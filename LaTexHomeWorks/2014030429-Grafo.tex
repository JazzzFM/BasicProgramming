\documentclass[12pt, A4]{article}
\usepackage[utf8]{inputenc}
\usepackage{amsmath}
\usepackage{amsfonts}
\usepackage{amssymb}
\usepackage{listings}
\usepackage{makeidx}
\usepackage{graphicx}
\usepackage[x11names,table]{xcolor}
\usepackage[spanish, es-tabla]{babel}
\usepackage{booktabs}
\usepackage{longtable,multirow,booktabs}
\renewcommand{\tablename}{Tabla}
\usepackage{hyperref}
\renewcommand{\refname}{Bibliograf\'ia}

\usepackage[spanish]{babel}
\usepackage[letterpaper]{geometry}
\usepackage[utf8]{inputenc}
\author{Jaziel David Flores Rodríguez.}
\date{\today}

\title{Grafos.} 
\begin{document}
	\maketitle

\section{Introducción.}
 Este documento contiene la descripción de grafo y su representación mediante su matriz de adyacencia; también se describe la forma de saber si un grafo tiene un recorrido euleriano y, de ser el caso, se indica  el recorrido euleriano.


\section{Descripción de un grafo}
Un grafo G agrupa entes físicos o conceptuales y las relaciones entre ellos. Un grafo está formado por un conjunto de vértices o nodos V , que representan a los entes, y un conjunto de arcos A ,
que representan las relaciones entre vértices. Se representa con el par G = (V, A).\\
Un arco o arista representa una relación entre dos nodos. Esta relación, al estar formada
por dos nodos, se representa por (u, v) siendo u, v el par de nodos. El grafo es no dirigido si
los arcos están formados por pares de nodos no ordenados, no apuntados; se representa con un
segmento uniendo los nodos, u — v .\\
Un grafo es dirigido, también denominado digrafo, si los pares de nodos que forman los arcos
son ordenados; se representan con una flecha que indica la dirección de la relación, u $\rightarrow$ v.\\
Dado el arco (u,v) de un grafo, se dice que los vértices u y v son adyacentes. Si el grafo es
dirigido, el vértice u es adyacente a v , y v es adyacente de u . En resumen se tiene la siguiente definición:\\\\
\textbf{Definición: } Un grafo permite modelar relaciones arbitrarias entre objetos. Un grafo G = (V,A) es un
par formado por un conjunto de vértices o nodos, V , y un conjunto de arcos o aristas, A .
Cada arco es el par (u,w) , siendo u, w dos vértices relacionados.\\Dado el arco (u,v) de un grafo, se dice que los vértices u y v son adyacentes. Si el grafo es
dirigido, el vértice u es adyacente a v , y v es adyacente de u .
\section{Representación de un grafo mediante su matriz de adyacencia}
La característica mas importante de un grafo, que distingue a uno de otro, es el conjunto de
pares de vértices que están relacionados, o que son adyacentes. Por ello, la forma más sencilla
de representación es mediante una matriz, de tantas filas/columnas como nodos, que permite
modelar fácilmente esa cualidad.\\
Sea G = (V, A) un grafo de n nodos, siendo V = $\{v_{0},v_{1},\ldots,v_{n-1}\}$ el conjunto de
nodos, y A = $\{(v_{i},v_{j})\}$ el conjunto de arcos. Los nodos están numerados consecutivamente de
$0$ a $n-1$. La representación de los arcos se hace con una matriz A de $n \times n$ elementos, denominada matriz de adyacencia, tal que todo elemento $a_{ij}$ puede tomar los valores:

\[a_{ij}=\left\{\begin{array}{lcc}
1&\mbox{si hay un arco}& (v_{i},v_{j})\\\\
0&\mbox{si no hay un arco}& (v_{i},v_{j})\\
\end{array}\right.\]

En los grafos no dirigidos la matriz de adyacencia siempre es simétrica ya que las relaciones entre
vértices no son ordenadas: si $v_{i}$ está relacionado con $v_{j}$ , entonces $v_{j}$ está relacionado con $v_{i}$.\\
Los grafos que modelan problemas en los que un arco tiene asociado una magnitud, un factor
de peso, también se representan mediante una matriz de tantas filas/columnas como nodos. Ahora
un elemento cualquiera, $a_{ij}$ representa el coste o factor de peso del arco $(v_{i} , v_{j} )$. Un arco que
no existe se puede representar con un valor imposible, por ejemplo infinito; también, se puede
representar con el valor $0$ , dependiendo de que $0$ no pueda ser un factor de peso significativo de
un arco. A esta matriz también se la denomina matriz valorada.
\section{Determinación si un grafo tiene un camino euleriano}
Un camino euleriano es un camino que pasa por cada arista una y solo una vez. Un ciclo o un circuito euleriano es un camino que recorre cada arista exactamente una vez.  
El siguiente programa determina si un grafo tiene recorrido euleriano y si es que lo tiene indica el recorrido. 
\begin{lstlisting}
import java.util.ArrayList;

public class Grafo {

private int vertices;
private ArrayList<Integer>[] adj;

Grafo(int NumeroDeVertices) {
this.vertices = NumeroDeVertices;
inicializar();
}

private void inicializar() {
adj = new ArrayList[vertices];
for (int i = 0; i < vertices; i++) {
adj[i] = new ArrayList<>();
}
}

private void agregarBorde(Integer u, Integer v) {
adj[u].add(v);
adj[v].add(u);
}

private void quitarBorde(Integer u, Integer v) {
adj[u].remove(v);
adj[v].remove(u);
}

private void imprimirCaminoEuler() {
Integer u = 0;
for (int i = 0; i < vertices; i++) {
if (adj[i].size() % 2 == 1) {
u = i;
break;
}
}
imprimir(u);
System.out.println();
}

private void imprimir(Integer u) {
for (int i = 0; i < adj[u].size(); i++) {
Integer v = adj[u].get(i);
if (esValidoSigBorde(u, v)) {
System.out.print(u + "-" + v + " ");
quitarBorde(u, v);
imprimir(v);
}
}
}

private boolean esValidoSigBorde(Integer u, Integer v) {
if (adj[u].size() == 1) {
return true;
}
boolean[] esVisitado = new boolean[this.vertices];
int contador1 = dfscontador(u, esVisitado);
quitarBorde(u, v);
esVisitado = new boolean[this.vertices];
int contador2 = dfscontador(u, esVisitado);
agregarBorde(u, v);
return (contador1 > contador2) ? false : true;
}

private int dfscontador(Integer v, boolean[] esVisitado) {
esVisitado[v] = true;
int contador = 1;
for (int adj : adj[v]) {
if (!esVisitado[adj]) {
contador = contador + dfscontador(adj, esVisitado);
}
}
return contador;
}

public static void main(String[] args) {
Grafo g1 = new Grafo(4);
g1.agregarBorde(0, 1);
g1.agregarBorde(0, 2);
g1.agregarBorde(1, 2);
g1.agregarBorde(2, 3);
g1.imprimirCaminoEuler();

Grafo g2 = new Grafo(3);
g2.agregarBorde(0, 1);
g2.agregarBorde(1, 2);
g2.agregarBorde(2, 0);
g2.imprimirCaminoEuler();

Grafo g3 = new Grafo(5);
g3.agregarBorde(1, 0);
g3.agregarBorde(0, 2);
g3.agregarBorde(2, 1);
g3.agregarBorde(0, 3);
g3.agregarBorde(3, 4);
g3.agregarBorde(3, 2);
g3.agregarBorde(3, 1);
g3.agregarBorde(2, 4);
g3.imprimirCaminoEuler();
}

}

\end{lstlisting}

\end{document}
